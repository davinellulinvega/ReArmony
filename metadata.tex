% DO NOT EDIT - automatically generated from metadata.yaml

\def \codeURL{https://gitlab.com/davinellulinvega/armony}
\def \codeDOI{}
\def \codeSWH{}
\def \dataURL{https://doi.org/10.5281/zenodo.5550570}
\def \dataDOI{10.5281/zenodo.5550570}
\def \editorNAME{}
\def \editorORCID{}
\def \reviewerINAME{}
\def \reviewerIORCID{}
\def \reviewerIINAME{}
\def \reviewerIIORCID{}
\def \dateRECEIVED{01 November 2018}
\def \dateACCEPTED{}
\def \datePUBLISHED{}
\def \articleTITLE{[Re] An anatomically constrained neural network model of fear conditioning}
\def \articleTYPE{Replication}
\def \articleDOMAIN{Computational Neuroscience}
\def \articleBIBLIOGRAPHY{bibliography.bib}
\def \articleYEAR{2024}
\def \reviewURL{}
\def \articleABSTRACT{This article presents a replication of the computational model and classical conditioning experiment initially published in the article entitled: "An anatomically constrained neural network model of fear conditioning". In the original paper, the author identified a set of principles underlying the mechanisms for transmitting information from the thalamus to the amygdala as suggested by the theory of the "Two-pathways to the amygdala". Through a simulated classical conditioning experiment, the author's goal is to validate the design principles by showing that the computational model is able to account for sets of physiological and behavioral findings from conditioning studies on animals. A second aim of the original article was to emphasize the usefulness of computer modeling as a tool to help neuroscience move forward. Here I show that I was able to re-implement the computational model described by Armony et al. (1995), as well as replicate the results initially published. Individual units within the model exhibited differences in activation after the conditioning phase which is consistent with empirical results from cell recording studies. Similarly, the behavioral output of the network matches experimental results from the classical conditioning paradigm. Therefore, the principles on which the computational model is built are valid. Moreover, the model has since proved its usefulness by being able to predict the outcome of lesion studies on the parallel pathways used by the brain to carry stimuli from the thalamus to the amygdala. }
\def \replicationCITE{Armony, J. L., Servan-Schreiber, D., Cohen, J. D., & LeDoux, J. E. (1995). An anatomically constrained neural network model of fear conditioning. Behavioral Neuroscience, 109(2), 246–257. https://doi.org/10.1037/0735-7044.109.2.246}
\def \replicationBIB{Armony1995}
\def \replicationURL{http://www.ncbi.nlm.nih.gov/pubmed/7619315}
\def \replicationDOI{10.1037/0735-7044.109.2.246}
\def \contactNAME{Luc Caspar}
\def \contactEMAIL{casparluc@gmail.com}
\def \articleKEYWORDS{Computational Neuroscience, Classical Conditioning, Python}
\def \journalNAME{ReScience C}
\def \journalVOLUME{4}
\def \journalISSUE{1}
\def \articleNUMBER{}
\def \articleDOI{}
\def \authorsFULL{Luc Caspar and Roger K. Moore}
\def \authorsABBRV{L. Caspar and R.K. Moore}
\def \authorsSHORT{Caspar and Moore}
\title{\articleTITLE}
\date{}
\author[1,\orcid{0000-0002-8518-9433}]{Luc Caspar}
\author[1,\orcid{0000-0003-0065-3311}]{Roger K. Moore}
\affil[1]{The University of Sheffield, Western Bank, Sheffield, S10 2TN, United Kingdom}
